\section{Introduction}
\label{sec:intro}
The integration of the Internet-Of-Things (IoT) devices within the residential environment has become increasingly prevalent over the past two decades. Everyday consumers are aware of the vast amount of automation that can be acquired through devices carrying wireless sensors and monitors. Many of these devices seek to provide increased home security through cameras, audio collection,  and alarms. Moreover, the connection of these various devices creates an ever-growing web of wireless communication. However, as the pervasiveness of IoT devices was estimated to be 50 billion devices in 2020 \cite{Helen}, risk is being introduced as devices attach to these often-insecure networks. 
    
The issue HomeSnitch attempts to solve is exactly this: a compromised or misbehaving smart home device may attack other network hosts, but it may also eavesdrop on the physical home environment \cite{Enck}. The challenge with this problem is identifying when a devices behavior is acceptable and when it has become malicious. For the majority of devices, it is not necessary to increase security when the device is in use, but rather monitor potential eavesdropping activities when the device is idle. Simple solutions such as turning a device off may seem sufficient, however, a long-term solution to access control is necessary to enhance network security. 

HomeSnitch functions as a building block for improving transparency and control over the network communications between smart home devices. This solution provides an ability to classify and control semantic behaviors using features that represent the application-layer dialogue between clients and servers \cite{Enck}. Many smart home devices have poor authentication mechanisms and software vulnerabilities that present threats due to unexpected behavior implemented by manufacturers. Overcoming the challenge of diagnosing device behavior while respecting encrypted payload data is crucial to providing transparency and control for the end user.  

The remainder of this paper proceeds as follows: Section 2 details the HomeSnitch model and the threat it attempts to solve. Section 3 details the techniques HomeSnitch utilizes to solve this problem. Section 4 provides an evaluation of experimental results. Section 5 offers a discussion of the limitations of the study and potential future research. Section 6 concludes. 